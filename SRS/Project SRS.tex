\begin{document}
Introduction:
	1.1 Purpose of this Document:
		This document will be a detailed analysis of the project I will be working during the summer. 
	1.2 Scope of this document:
	1.3 Overview:
			The program is mainly for college students or those just entering college and need to create a schedule for their classes.  The program will allow students to create effective schedules for their classes without becoming a burden.  Also students will be able to fit their courses around their own day to day schedule, instead of fitting their schedule around their courses.
	1.4 Business Context:
		AMP Program��.
General Description:
	2.1 Product Functions:
		The system should be able to take in multiple classes with different time slots, create a unique schedule out of every possible combination of classes, then display the schedules that have no conflicting time slots.  The system should also be able to take in other schedules, such as work or any times the student would not like to have class, and determine whether it will conflict with the course schedule.
The system should make it effortless for GMU students to create class schedules that will not interfere with their daily lives.  The system should not jeopardize ease of use for performance.
	2.2 Similar System information:
		The product will be used along side a website that will hold the system.  The system will be a web applet. If we can finish the project before the deadline we might be able to turn it into a web application.
	2.3 User Characteristics:
		Users do not need prior experience to run the software, but should be familiar with how scheduling works and day to day applications.
	2.4 User Problem Statement:
		In order to use the software, users will need to have prepared the classes they would like to schedule which might be difficult.  Users will have to input every single potential timeslot which might become time consuming.
	2.5 User Objectives:
		Users will want an easy and fast way to input their courses and the individual timeslots.  Inputting so many different timeslots can become very tedious, so the GUI would need to lessen this task.
We can do this by asking for the class name once, then create textboxes for the users to input different times.  A combo box might be best for the minutes.
Finally the schedule should be displayed in a manner that is easy to read and comprehend.  Somewhat like a calendar.
	2.6 General Constraints:
		Since the program will be a web applet it will not have access to system resources, so we cannot save files or run programs on the user�s computer.  We will also need to make the program reusable incase we want to turn it into a stand alone application or a web application.


Functional Requirements:

	System must be able to compute working schedules efficiently and in a timely manner.
		Description:
			The applet should not take long to load the working schedules and display them to the user.  
		Criticality:
			Very critical
		Technical issues:
			System will have to not only compare each course but also its labs.  It will also have to compare the final schedules with other the timeslots the student would not like to have class.
		Risks:
			If the user inputs wrong data, such as replacing the start time of a course with its end time, the system will not be able to process it.  

	System should be able to store the data temporarily.
		Description:	
			If the user wants to go back and enter another class or change a timeslot, he should not have to enter all the data in again.  The user should be able to go back to his previous courses easily.
		Criticality:
			Very critical.
		Technical issues:
			The system will have to somehow display the current courses and timeslots in a way that is easy for the user to read and change.
		Risks:
			none

	GUI should display the final results in a manner that is readable.
		Description:
			After the program calculates the working schedules, it should display a chart containing a schedule.  As the user looks through the schedules the chart should redisplay the current schedule that the user is looking at.
			The chart that is displayed should highlight the classes and the times that the user listed as not wanted.
		Criticality:
			Very Critical.
		Technical issues:
			I am very bad with GUI designs.
		Risks:
			Fitting the chart with a schedule that has a lot of classes, in a predetermined size box.

	The program should ask for classes that are optional for the student.
		Description:
			Some classes students are not sure whether they should take them or not depending on their schedule.  The system should show any schedule that contains or does not contain the optional classes.
		Criticality:
			Somewhat critical.
		Technical issues:
			Applet should have a check box, that users can check if the class is optional.
		Risks:
			If there are too many optional classes it will greatly increase the time it takes to calculate the resulting schedules.

Interface Requirements:
	4.1 User Interfaces:
		The system will interface with the user mainly through the GUI.
	4.1.1 GUI:
		The graphical user interface will first present the user with a couple of text boxes and combo boxes for the user to input the class name and its timeslots.  It will also have 4 buttons, NEXT, PREVIOUS, REMOVE, and ADD, which will add that class to the list.
			This will also contain check boxes next to the classes.  The user can check these if the class is optional. 
			It will also have a PANEL that displays the classes that were added.  The user can click on any of these to highlight a class, and change its properties.
		The next phase, the GUI will present the user with similar boxes for the user to input when he would not be available for classes. 
			This phase might have a chart that the user can use to enter when he is unavailable instead of textboxes.
		The third phase, will display the resulting schedule to the student.  It will have a PREVIOUS button incase the student wants to change any class.
	4.1.2 API:
		To be filled later.

Performance Requirements:
	Since the system will be an applet the user will only need an internet connection.  If we make it a stand alone application, then the user will need a java virtual machine.

Design Constraints:
	If the system will save user�s schedules for later use, the information cannot be shown to any other users. 

Other Non-Functional Attributes:
\end{document}
